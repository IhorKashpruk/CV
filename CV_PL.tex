\documentclass[12pt,a4paper,sans]{moderncv}
\usepackage[utf8]{inputenc}
\moderncvstyle{banking}
\moderncvcolor{blue}
\usepackage[scale=0.75]{geometry}

\newcommand{\changefont}{%
    \fontsize{9}{12}\selectfont
}

\name{Ihor}{Kashpruk}
\address{ul. S. Dubois 22/3, Wrocław, Polska, 52-007}
\phone[mobile]{+48 732 802 641}
\email{igork1823@gmail.com}
\social[github][www.github.com]{IhorKashpruk}

\title{Curriculum Vitae}

\begin{document}
\makecvtitle
\small{Absolwent studiów informatycznych. Pasjąnuję się programowaniem, posiadam silne umiejętnośći techniczne, interpersonalne. Ciągłe się rozwijam w poznawaniu nowych technologii.}

\section{Wykształcenie}

\begin{itemize}
\item{\cventry{2010--2014}{wyższe (młodszy inżynier), Informatyka}{Technikum od Nardowego Uniwersytetu Technologii Żywnośći}{Winnica, Ukraina}{}{}}
\item{\cventry{2014--2018}{wyższe (inżynier), Informatyka, inżyneria oprogramowania}{Wyższa szkoła Ekonomii i Innowacji}{Lublin, Polska}{}{}}

\end{itemize}

\section{Technologie}
\begin{itemize}

\item{C/C++(bardzo dobry), QT/QML, STL, multithreading (MPI, OpenMP, std::thread, pthread), SDL2, Box2D, network programing(Boost.Asio, TCP/IP socket programming in C)}
\item{Java, JavaFX}
\item{C\#}
\item{Python}
\item{JavaScript, HTML, CSS}
\item{SQL, MySQL}
\item{GitHub}
\end{itemize}

\section{Umiejętnośći}
\begin{itemize}
\item{Linux(preferowany), Windows}
\item{programowanie siecowe, systemowe(Linux) ta metaprogramowanie}
\item{wielowątkowość}
\item{OOP, wzorce projektowe, UML}
\item{podstawy baz danych, podstawy testowania}
\end{itemize}

\section{Projekty}
GitHub:
\textcolor{blue}{\url{https://github.com/IhorKashpruk}}

\section{Języki obce}
\begin{itemize}

\item{Polski - zaawansowany}
\item{Angielski - średnio zaawansowany}
\item{Rosyjski, Ukraiński - biegły}
\end{itemize}

\fancyfoot[C]{\changefont \textcolor{gray}{"Wyrażam zgodę na przetwarzanie moich danych osobowych dla potrzeb niezbędnych do realizacji procesu rekrutacji (zgodnie z ustawą z dnia 10 maja 2018 roku o ochronie danych osobowych (Dz. Ustaw z 2018, poz. 1000) oraz zgodnie z Rozporządzeniem Parlamentu Europejskiego i Rady (UE) 2016/679 z dnia 27 kwietnia 2016 r. w sprawie ochrony osób fizycznych w związku z przetwarzaniem danych osobowych i w sprawie swobodnego przepływu takich danych oraz uchylenia dyrektywy 95/46/WE (RODO)."}}
\end{document}
